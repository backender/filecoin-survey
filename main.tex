
%% This is a skeleton file demonstrating the use of IEEEtran.cls
%% (requires IEEEtran.cls version 1.7 or later) with an IEEE journal paper.
%%
%% Support sites:
%% http://www.michaelshell.org/tex/ieeetran/
%% http://www.ctan.org/tex-archive/macros/latex/contrib/IEEEtran/
%% and
%% http://www.ieee.org/


\documentclass[journal]{IEEEtran}
\usepackage{blindtext}
\usepackage{graphicx}
\usepackage{url}


% *** GRAPHICS RELATED PACKAGES ***
%
\ifCLASSINFOpdf
  % \usepackage[pdftex]{graphicx}
  % declare the path(s) where your graphic files are
  % \graphicspath{{../pdf/}{../jpeg/}}
  % and their extensions so you won't have to specify these with
  % every instance of \includegraphics
  % \DeclareGraphicsExtensions{.pdf,.jpeg,.png}
\else
  % or other class option (dvipsone, dvipdf, if not using dvips). graphicx
  % will default to the driver specified in the system graphics.cfg if no
  % driver is specified.
  % \usepackage[dvips]{graphicx}
  % declare the path(s) where your graphic files are
  % \graphicspath{{../eps/}}
  % and their extensions so you won't have to specify these with
  % every instance of \includegraphics
  % \DeclareGraphicsExtensions{.eps}
\fi
% graphicx was written by David Carlisle and Sebastian Rahtz. It is
% required if you want graphics, photos, etc. graphicx.sty is already
% installed on most LaTeX systems. The latest version and documentation can
% be obtained at: 
% http://www.ctan.org/tex-archive/macros/latex/required/graphics/
% Another good source of documentation is "Using Imported Graphics in
% LaTeX2e" by Keith Reckdahl which can be found as epslatex.ps or
% epslatex.pdf at: http://www.ctan.org/tex-archive/info/
%
% latex, and pdflatex in dvi mode, support graphics in encapsulated
% postscript (.eps) format. pdflatex in pdf mode supports graphics
% in .pdf, .jpeg, .png and .mps (metapost) formats. Users should ensure
% that all non-photo figures use a vector format (.eps, .pdf, .mps) and
% not a bitmapped formats (.jpeg, .png). IEEE frowns on bitmapped formats
% which can result in "jaggedy"/blurry rendering of lines and letters as
% well as large increases in file sizes.
%
% You can find documentation about the pdfTeX application at:
% http://www.tug.org/applications/pdftex



% correct bad hyphenation here
\hyphenation{op-tical net-works semi-conduc-tor}


\begin{document}
%
% paper title
% can use linebreaks \\ within to get better formatting as desired
\title{Is filecoin a \$257 Million bubble or a pyramid scheme?}
%
%
% author names and IEEE memberships
% note positions of commas and nonbreaking spaces ( ~ ) LaTeX will not break
% a structure at a ~ so this keeps an author's name from being broken across
% two lines.
% use \thanks{} to gain access to the first footnote area
% a separate \thanks must be used for each paragraph as LaTeX2e's \thanks
% was not built to handle multiple paragraphs
%

\author{
\IEEEauthorblockN{Marc Juchli} \\
\IEEEauthorblockA{EEMCS,  
Delft University of Technology\\
m.b.juchli@student.tudelft.nl}   %<------ Line breaks in the current column
}

% The paper headers
%\markboth{The Culture Clash, September~2016}%
%{}
% The only time the second header will appear is for the odd numbered pages

% make the title area
\maketitle

\begin{abstract}
%\boldmath

...

\end{abstract}

\begin{IEEEkeywords}
technical writing, teaching, culture.
\end{IEEEkeywords}

\IEEEpeerreviewmaketitle

\section{Introduction}

"A MASSIVE AMOUNT OF STORAGE SITS UNUSED IN DATA CENTERS AND HARD DRIVES AROUND THE WORLD."
With this slogan Protocol Labs is about to disrupt the storage market by using \textit{proof-of-spacetime} as their driving source.
The Filecoin project describes a decentralized storage market where anyone, worldwide, is able to participate as a storage provider.
The concept is indeed promising and convinced the investors such that a total of \$257 million had been raised –-- the biggest initial coin offering (ICO) as of today (September 2017).

Decentralized storage market is not a novel concept.
Others[Tribler, TorCoin, Mojo Nation] have tried in past too, but yet were not able to scale as much as Filecoin advertises to do, and eventually failed.
More recent projects [StorJ, Swarm, Sia, MaidSAFE] are currently working towards building a similar system with conceptual differences which will be uncovered briefly in this paper.

This paper aims to analyze the ICO launch of Filecoin and reasons about the exceptionally large investment using heavily discussed topics in social media channels.
Further, the feasibility in terms of technical as well as economical design is studied.
We aim to uncover potential weaknesses but also highlight strengths of the proposed white-paper [Whitepaper].
The incorporated \textit{proof-of-spacetime} consensus algorithm is being highlighted and compared with consensus proposals from projects such as StorJ and Sia.
Eventually, the ICO launch is opposed to the scientific "1975 rule", a model of a pyramid scheme [Pyramid Scheme] and the feasibility is confronted with the "speculative bubble" model [Bubble].

The structure of this paper is as follows:

\section{15 years of documented failure}

\section{ICO analysis}

\subsection{Investors exclusion}

\section{Is the design technically feasible?}

\section{Is the design economically feasible?}

\section{Is this a pyramid scheme?}

\section{Is it a bubble?}


% Can use something like this to put references on a page
% by themselves when using endfloat and the captionsoff option.
\ifCLASSOPTIONcaptionsoff
  \newpage
\fi

% trigger a \newpage just before the given reference
% number - used to balance the columns on the last page
% adjust value as needed - may need to be readjusted if
% the document is modified later
%\IEEEtriggeratref{8}
% The "triggered" command can be changed if desired:
%\IEEEtriggercmd{\enlargethispage{-5in}}

% references section

% can use a bibliography generated by BibTeX as a .bbl file
% BibTeX documentation can be easily obtained at:
% http://www.ctan.org/tex-archive/biblio/bibtex/contrib/doc/
% The IEEEtran BibTeX style support page is at:
% http://www.michaelshell.org/tex/ieeetran/bibtex/
%\bibliographystyle{IEEEtran}
% argument is your BibTeX string definitions and bibliography database(s)
%\bibliography{IEEEabrv,../bib/paper}
%
% <OR> manually copy in the resultant .bbl file
% set second argument of \begin to the number of references
% (used to reserve space for the reference number labels box)
\begin{thebibliography}{1}

\bibitem{mit}Lec 1 | MIT 6.00 Introduction to Computer Science and Programming, Fall 2008, \url{https://www.youtube.com/watch?v=k6U-i4gXkLM}, Aug 19, 2009.

% Bubble: http://www.sciencedirect.com/science/article/pii/016517657990017X

\end{thebibliography}

% biography section
% 
% If you have an EPS/PDF photo (graphicx package needed) extra braces are
% needed around the contents of the optional argument to biography to prevent
% the LaTeX parser from getting confused when it sees the complicated
% \includegraphics command within an optional argument. (You could create
% your own custom macro containing the \includegraphics command to make things
% simpler here.)
%\begin{biography}[{\includegraphics[width=1in,height=1.25in,clip,keepaspectratio]{mshell}}]{Michael Shell}
% or if you just want to reserve a space for a photo:


% You can push biographies down or up by placing
% a \vfill before or after them. The appropriate
% use of \vfill depends on what kind of text is
% on the last page and whether or not the columns
% are being equalized.

%\vfill

% Can be used to pull up biographies so that the bottom of the last one
% is flush with the other column.
%\enlargethispage{-5in}



% that's all folks
\end{document}


